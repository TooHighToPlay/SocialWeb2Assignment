\documentclass{acm_proc_10ptArticle-sp}
\usepackage[T1]{fontenc}

\begin{document}

\title{Assignment 3 - Social Web}

\numberofauthors{3}
\author{
%
% The command \alignauthor (no curly braces needed) should
% precede each author name, affiliation/snail-mail address and
% e-mail address. Additionally, tag each line of
% affiliation/address with \affaddr, and tag the
%% e-mail address with \email.
\alignauthor Arthur-Ervin Avramiea\\
       \affaddr{2517642}\\
       \email{a.e.avramiea@student.vu.nl}
\alignauthor Mihnea Dobrescu-Balaur\\
	\affaddr{2549278}\\
	\email{mihnea@linux.com}
\alignauthor Zilvinas Kucinskas\\
	\affaddr{zks300}\\
        \email{zil.kucinskas@gmail.com}
}

\date{4 March 2014}
\maketitle

\section{Discuss an existing analysis}

what data is used

what analyses are performed

which types of analyses are performed


\section{New analysis}

Research hypothesis - for a topic related to a country - citizens from the country tend to get informed from local news. people from outside the country use more sources. (to find article) does this also hold for the new media?

Describe data collection

Describe data analysis performed

Describe results

- tweets by country

- percentage of local media focus

- main parties involved - which are the media sources

- media heterogeneity (to define a measure)

Limitations - the use of different country sources may reflect immigrants, especially when speaking less "popular" languages

\section{Conclusions} 

% The following two commands are all you need in the
% initial runs of your .tex file to
% produce the bibliography for the citations in your paper.
\bibliographystyle{abbrv}
\bibliography{assignment3}


\end{document}
